\documentclass[]{article}
\usepackage{lmodern}
\usepackage{amssymb,amsmath}
\usepackage{ifxetex,ifluatex}
\usepackage{fixltx2e} % provides \textsubscript
\ifnum 0\ifxetex 1\fi\ifluatex 1\fi=0 % if pdftex
  \usepackage[T1]{fontenc}
  \usepackage[utf8]{inputenc}
\else % if luatex or xelatex
  \ifxetex
    \usepackage{mathspec}
  \else
    \usepackage{fontspec}
  \fi
  \defaultfontfeatures{Ligatures=TeX,Scale=MatchLowercase}
\fi
% use upquote if available, for straight quotes in verbatim environments
\IfFileExists{upquote.sty}{\usepackage{upquote}}{}
% use microtype if available
\IfFileExists{microtype.sty}{%
\usepackage{microtype}
\UseMicrotypeSet[protrusion]{basicmath} % disable protrusion for tt fonts
}{}
\usepackage[margin=1in]{geometry}
\usepackage{hyperref}
\hypersetup{unicode=true,
            pdftitle={Bio-Sparte},
            pdfborder={0 0 0},
            breaklinks=true}
\urlstyle{same}  % don't use monospace font for urls
\usepackage{color}
\usepackage{fancyvrb}
\newcommand{\VerbBar}{|}
\newcommand{\VERB}{\Verb[commandchars=\\\{\}]}
\DefineVerbatimEnvironment{Highlighting}{Verbatim}{commandchars=\\\{\}}
% Add ',fontsize=\small' for more characters per line
\usepackage{framed}
\definecolor{shadecolor}{RGB}{248,248,248}
\newenvironment{Shaded}{\begin{snugshade}}{\end{snugshade}}
\newcommand{\AlertTok}[1]{\textcolor[rgb]{0.94,0.16,0.16}{#1}}
\newcommand{\AnnotationTok}[1]{\textcolor[rgb]{0.56,0.35,0.01}{\textbf{\textit{#1}}}}
\newcommand{\AttributeTok}[1]{\textcolor[rgb]{0.77,0.63,0.00}{#1}}
\newcommand{\BaseNTok}[1]{\textcolor[rgb]{0.00,0.00,0.81}{#1}}
\newcommand{\BuiltInTok}[1]{#1}
\newcommand{\CharTok}[1]{\textcolor[rgb]{0.31,0.60,0.02}{#1}}
\newcommand{\CommentTok}[1]{\textcolor[rgb]{0.56,0.35,0.01}{\textit{#1}}}
\newcommand{\CommentVarTok}[1]{\textcolor[rgb]{0.56,0.35,0.01}{\textbf{\textit{#1}}}}
\newcommand{\ConstantTok}[1]{\textcolor[rgb]{0.00,0.00,0.00}{#1}}
\newcommand{\ControlFlowTok}[1]{\textcolor[rgb]{0.13,0.29,0.53}{\textbf{#1}}}
\newcommand{\DataTypeTok}[1]{\textcolor[rgb]{0.13,0.29,0.53}{#1}}
\newcommand{\DecValTok}[1]{\textcolor[rgb]{0.00,0.00,0.81}{#1}}
\newcommand{\DocumentationTok}[1]{\textcolor[rgb]{0.56,0.35,0.01}{\textbf{\textit{#1}}}}
\newcommand{\ErrorTok}[1]{\textcolor[rgb]{0.64,0.00,0.00}{\textbf{#1}}}
\newcommand{\ExtensionTok}[1]{#1}
\newcommand{\FloatTok}[1]{\textcolor[rgb]{0.00,0.00,0.81}{#1}}
\newcommand{\FunctionTok}[1]{\textcolor[rgb]{0.00,0.00,0.00}{#1}}
\newcommand{\ImportTok}[1]{#1}
\newcommand{\InformationTok}[1]{\textcolor[rgb]{0.56,0.35,0.01}{\textbf{\textit{#1}}}}
\newcommand{\KeywordTok}[1]{\textcolor[rgb]{0.13,0.29,0.53}{\textbf{#1}}}
\newcommand{\NormalTok}[1]{#1}
\newcommand{\OperatorTok}[1]{\textcolor[rgb]{0.81,0.36,0.00}{\textbf{#1}}}
\newcommand{\OtherTok}[1]{\textcolor[rgb]{0.56,0.35,0.01}{#1}}
\newcommand{\PreprocessorTok}[1]{\textcolor[rgb]{0.56,0.35,0.01}{\textit{#1}}}
\newcommand{\RegionMarkerTok}[1]{#1}
\newcommand{\SpecialCharTok}[1]{\textcolor[rgb]{0.00,0.00,0.00}{#1}}
\newcommand{\SpecialStringTok}[1]{\textcolor[rgb]{0.31,0.60,0.02}{#1}}
\newcommand{\StringTok}[1]{\textcolor[rgb]{0.31,0.60,0.02}{#1}}
\newcommand{\VariableTok}[1]{\textcolor[rgb]{0.00,0.00,0.00}{#1}}
\newcommand{\VerbatimStringTok}[1]{\textcolor[rgb]{0.31,0.60,0.02}{#1}}
\newcommand{\WarningTok}[1]{\textcolor[rgb]{0.56,0.35,0.01}{\textbf{\textit{#1}}}}
\usepackage{graphicx,grffile}
\makeatletter
\def\maxwidth{\ifdim\Gin@nat@width>\linewidth\linewidth\else\Gin@nat@width\fi}
\def\maxheight{\ifdim\Gin@nat@height>\textheight\textheight\else\Gin@nat@height\fi}
\makeatother
% Scale images if necessary, so that they will not overflow the page
% margins by default, and it is still possible to overwrite the defaults
% using explicit options in \includegraphics[width, height, ...]{}
\setkeys{Gin}{width=\maxwidth,height=\maxheight,keepaspectratio}
\IfFileExists{parskip.sty}{%
\usepackage{parskip}
}{% else
\setlength{\parindent}{0pt}
\setlength{\parskip}{6pt plus 2pt minus 1pt}
}
\setlength{\emergencystretch}{3em}  % prevent overfull lines
\providecommand{\tightlist}{%
  \setlength{\itemsep}{0pt}\setlength{\parskip}{0pt}}
\setcounter{secnumdepth}{0}
% Redefines (sub)paragraphs to behave more like sections
\ifx\paragraph\undefined\else
\let\oldparagraph\paragraph
\renewcommand{\paragraph}[1]{\oldparagraph{#1}\mbox{}}
\fi
\ifx\subparagraph\undefined\else
\let\oldsubparagraph\subparagraph
\renewcommand{\subparagraph}[1]{\oldsubparagraph{#1}\mbox{}}
\fi

%%% Use protect on footnotes to avoid problems with footnotes in titles
\let\rmarkdownfootnote\footnote%
\def\footnote{\protect\rmarkdownfootnote}

%%% Change title format to be more compact
\usepackage{titling}

% Create subtitle command for use in maketitle
\providecommand{\subtitle}[1]{
  \posttitle{
    \begin{center}\large#1\end{center}
    }
}

\setlength{\droptitle}{-2em}

  \title{Bio-Sparte}
    \pretitle{\vspace{\droptitle}\centering\huge}
  \posttitle{\par}
    \author{}
    \preauthor{}\postauthor{}
    \date{}
    \predate{}\postdate{}
  

\begin{document}
\maketitle

\hypertarget{libraries}{%
\subsubsection{Libraries}\label{libraries}}

Library aufrufen das wir nutzen wollen

\begin{verbatim}
## -- Attaching packages --------------------------------------- tidyverse 1.2.1 --
\end{verbatim}

\begin{verbatim}
## v ggplot2 3.2.1     v purrr   0.3.2
## v tibble  2.1.3     v dplyr   0.8.3
## v tidyr   1.0.0     v stringr 1.4.0
## v readr   1.3.1     v forcats 0.4.0
\end{verbatim}

\begin{verbatim}
## -- Conflicts ------------------------------------------ tidyverse_conflicts() --
## x dplyr::filter() masks stats::filter()
## x dplyr::lag()    masks stats::lag()
\end{verbatim}

\begin{verbatim}
## Loading required package: RPostgreSQL
\end{verbatim}

\hypertarget{connection}{%
\subsubsection{Connection}\label{connection}}

Datanbankverbindung aufbauen

\hypertarget{grabbing-the-data}{%
\subsubsection{Grabbing the data}\label{grabbing-the-data}}

Abrufen von daten um damit umzugehen

\hypertarget{control-missing-data-na}{%
\paragraph{Control missing data (NA)}\label{control-missing-data-na}}

Kontrolle ob es fehlende werte gibt in der tabelle t\_aisles

\hypertarget{aisle}{%
\subparagraph{Aisle}\label{aisle}}

\begin{verbatim}
## [1] 0
\end{verbatim}

\begin{verbatim}
## [1] 0
\end{verbatim}

\hypertarget{departments}{%
\subparagraph{Departments}\label{departments}}

\begin{verbatim}
## [1] 0
\end{verbatim}

\begin{verbatim}
## [1] 0
\end{verbatim}

\hypertarget{orders}{%
\subparagraph{Orders}\label{orders}}

\begin{verbatim}
## [1] 0
\end{verbatim}

\begin{verbatim}
## [1] 0
\end{verbatim}

\begin{verbatim}
## [1] 0
\end{verbatim}

\begin{verbatim}
## [1] 0
\end{verbatim}

\begin{verbatim}
## [1] 0
\end{verbatim}

\begin{verbatim}
## [1] 0
\end{verbatim}

Kunden die zum ersten mal einkaufen

Es fehlen 206209 werte von 3421083 für days\_since\_prior (6.0275942\%.

Es könnten first-time Kunden sein, also die Kunden die ihre erste
Bestellung machen

\hypertarget{products}{%
\subparagraph{Products}\label{products}}

\begin{verbatim}
## [1] 0
\end{verbatim}

\begin{verbatim}
## [1] 0
\end{verbatim}

Jetzt wollen wir graphisch das verhältnis von Bio- zu nicht Bio-Produkte
aufzeichnen

\includegraphics{Bio-Sparte_files/figure-latex/unnamed-chunk-7-1.pdf}

\hypertarget{anzahl-gekaufte-bio-produkte-in-respekt-zu-nicht-bio-produkte}{%
\subsubsection{Anzahl Gekaufte Bio-Produkte in respekt zu nicht
Bio-Produkte}\label{anzahl-gekaufte-bio-produkte-in-respekt-zu-nicht-bio-produkte}}

\begin{Shaded}
\begin{Highlighting}[]
\NormalTok{Verhalt_Bio_bought <-}\StringTok{ }\NormalTok{t_products}
\end{Highlighting}
\end{Shaded}

\hypertarget{verhuxe4ltnis-bionicht-bio-in-departments-aisles}{%
\subsubsection{Verhältnis Bio/Nicht-bio in departments \&
aisles}\label{verhuxe4ltnis-bionicht-bio-in-departments-aisles}}

Wir möchten auch Visualizieren was die Verhältnisse sind, zwischen die
Offerte an Bio-produkte in den verschiedene Departments und Aisles.

\includegraphics{Bio-Sparte_files/figure-latex/unnamed-chunk-9-1.pdf}

Man kann sehen das die Anzahl an Bio-Produkte in den Departments sehr
swach ist, im gegensatz zu den ``normalen'' Produkten.

\includegraphics{Bio-Sparte_files/figure-latex/unnamed-chunk-10-1.pdf}

Man kann auch hier sehen das der Verhältniss zwischen Bio-Produkten und
nicht Bio\_Produkten niedrig ist in spezifische aisles. Es gibt sogar
aisles wo es keinlerei Bio-Produkten gibt. Dafür aber gibt es andere
aisles wo der Verhältnis viel grösser ist, z.b. Baby Food Formulas und
Fresh Vegetables.

\includegraphics{Bio-Sparte_files/figure-latex/unnamed-chunk-11-1.pdf}

\hypertarget{determine-most-bought-products}{%
\subsubsection{Determine most bought
products}\label{determine-most-bought-products}}

Wir möchten wissen welche Produkte am meisten eingekauft werden und ob
diese Organic oder nicht sind

\begin{verbatim}
## # A tibble: 30 x 3
## # Groups:   organic [2]
##    organic product_name               n
##    <lgl>   <chr>                  <int>
##  1 FALSE   Banana                 18726
##  2 TRUE    Bag of Organic Bananas 15480
##  3 TRUE    Organic Strawberries   10894
##  4 TRUE    Organic Baby Spinach    9784
##  5 FALSE   Large Lemon             8135
##  6 TRUE    Organic Avocado         7409
##  7 TRUE    Organic Hass Avocado    7293
##  8 FALSE   Strawberries            6494
##  9 FALSE   Limes                   6033
## 10 TRUE    Organic Raspberries     5546
## # ... with 20 more rows
\end{verbatim}

\includegraphics{Bio-Sparte_files/figure-latex/unnamed-chunk-12-1.pdf}

\includegraphics{Bio-Sparte_files/figure-latex/unnamed-chunk-13-1.pdf}

\includegraphics{Bio-Sparte_files/figure-latex/unnamed-chunk-14-1.pdf}

\hypertarget{stammkunden-analyse}{%
\subsection{Stammkunden analyse}\label{stammkunden-analyse}}

Ein ziele wäre die Stammkunden erkennen und sehen ob sie verschieden
einkaufen, z.b eine höhere Anzahl an Produkten pro Bestellung kaufen.
Daher haben wir der durschnitt an Bestellungen berechnet sowie die
Quartilen: Ausser dem Visualizieren wir mit einen Plot die Anzahl kunden
die Mehrmals bei uns eingekauft haben.

\begin{verbatim}
## # A tibble: 100 x 2
##    nr_of_orders      n
##           <int>  <int>
##  1            1 206209
##  2            2 206209
##  3            3 206209
##  4            4 206209
##  5            5 182223
##  6            6 162633
##  7            7 146468
##  8            8 132618
##  9            9 120918
## 10           10 110728
## # ... with 90 more rows
\end{verbatim}

\includegraphics{Bio-Sparte_files/figure-latex/unnamed-chunk-15-1.pdf}

\begin{verbatim}
## [1] 17.15486
\end{verbatim}

\begin{verbatim}
##   0%  25%  50%  75% 100% 
##    1    5   11   23  100
\end{verbatim}

Wir wollen auch den Durschnitt von Bestellung von jeden Kunden
Berechnen.

\includegraphics{Bio-Sparte_files/figure-latex/unnamed-chunk-16-1.pdf}

Wir definieren Stammkunden die die mindestens einmal alle 14 Tagen bei
uns einkaufen (Alle 14 Tage, im diagramm 0 bis 13)

Die totale anzahl an stammkunden ist 103005 von 206209. Dies entsprecht
49.951748\% von aller Kunden die je im laden waren

\includegraphics{Bio-Sparte_files/figure-latex/unnamed-chunk-18-1.pdf}
Die Anzahl an transactionen das von Stammkunden geamcht wurden
entsprechen 702319 von 1384617. Das entspricht 50.7229797\% aller
transactionen.

\hypertarget{a-priori-algorithmus}{%
\subsection{A priori Algorithmus}\label{a-priori-algorithmus}}


\end{document}
